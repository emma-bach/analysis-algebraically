\iffalse
\DocumentMetadata{
	lang        = de,
	pdfstandard = ua-2,
	pdfstandard = a-4f, %or a-4
	tagging=on,
	tagging-setup={math/setup=mathml-SE} 
}
\fi
\documentclass{report}

% custom margins
\usepackage[a4paper,margin=1.5in]{geometry}
\renewcommand{\baselinestretch}{1.2}
%\AddToHook{cmd/section/before}{\clearpage}

% emma's long list of custom macros and universally used packages
\include{./macros-and-packages-en.tex}

% colored box behind proofs
\tcolorboxenvironment{proof}{
	colback=white,
	boxrule=0pt,
	frame hidden,
	borderline west={1pt}{0pt}{black},
	before skip=0.75cm,
	after skip=0.75cm,
	sharp corners,
	breakable,
	enhanced,
}

\renewenvironment{proofsketch}{\begin{proof}[Proof Sketch]\renewcommand*{\qedsymbol}{\("\square"\)}}{\end{proof}}

\pagestyle{fancy} %allows headers

\lhead{Emma Yoneda}
\rhead{\today}

\begin{document}
	
	\begin{titlepage}
	\centering
	{\huge\bfseries Analysis for Algebraists\par}
    \vspace{0.5cm}
	{\Large\itshape Emma Yoneda\par}
	\vfill
	

% Bottom of the page
	{\large \today\par}
\end{titlepage}

	\tableofcontents
	\thispagestyle{fancy}
	\chapter{Sets and Orders}
	\chapter{Ordered Fields}
		\iffalse
		\begin{definition}
			An \tbf{ordered commutative Ring} $(R,P)$ is a commutative Ring $K$ equipped with a set $P$ of \tbf{positive Elements}, such that:
			\begin{enumerate}
				\item For every $x \in K$, we have exactly one of $x \in P$ ($x$ is \tit{positive}), $x = 0$, or $-x \in P$ ($x$ is \tit{negative}).
				\item If $x$ and $y$ are positive, then so are $x + y$ and $x \cdot y$.
			\end{enumerate}
		\end{definition}
		We say that $K$ is \tit{ordered by $P$}.
		\fi
		\begin{definition}
		An \tbf{ordered commutative Ring} $(R, \leq)$ is a commutative Ring $R$ equipped with an \tbf{ordering relation} $\leq$, such that for all $a,b,c \in R$, we have:
		\begin{enumerate}
			\item $\leq$ defines a \tit{total order} on $F$. i.e:
			\begin{enumerate}
				\item $a \leq a$ (the order is \tit{reflexive}),
				\item $a \leq b \wedge b \leq c \implies a \leq c$ (the order is \tit{transitive}),
				\item $a \leq b \wedge b \leq a \implies a = b$ (the order is \tit{antisymmetric}),
				\item $a \leq b \vee b \leq a$ (the order is \tit{strongly connected})
			\end{enumerate}
			\item $a \leq b \implies a + c \leq b + c$
			\item $0 \leq a \wedge 0 \leq b \implies 0 \leq ab$
		\end{enumerate}
		\end{definition}
				\begin{lemma}
			For every ordered commutative Ring $(R, \leq)$ and $a \in R$, we have $-a \leq 0 \leq a$ or $a \leq 0 \leq -a$.
		\end{lemma}
		\begin{proof}
			Since the order $\leq$ is strongly connected, we have $a \leq 0$ or $0 \leq a$.
			\begin{enumerate}
				\item If $a \leq 0$, then we have $-a + a \leq 0 + -a$, i.e. $a \leq 0 \leq -a$,
				\item if $0 \leq a$, then we have $-a + 0 \leq -a + a$, i.e. $-a \leq 0 \leq a$
			\end{enumerate}
		\end{proof}
		\begin{lemma}
			Let $a \in (R, \leq)$. Then $0 \leq a^2$.
		\end{lemma}
		\begin{proof}
			Since the order $\leq$ is strongly connected, we have $a \leq 0$ or $0 \leq a$.
			\begin{enumerate}
				\item If $0 \leq a$, then we have $0 \leq a \cdot a = a^2$,
				\item if $a \leq 0$, then $0 \leq -a$ and we have $0 \leq -a \cdot (-a) = a^2$.
			\end{enumerate}
			 
		\end{proof}
		\begin{lemma}
			Every ordered commutative Ring has characteristic $0$.
		\end{lemma}
		\begin{proof}
			Assume that $F$ is a field of characteristic $p$. Then an ordering relation would need to fulfill:
			\begin{align*}
				1 \leq 1 + 1 \leq \sum_{i = 1}^p 1 = 0 \leq 1
			\end{align*}
			Which implies $0 = 1$. However, by the definition of a field, we have $0 \neq 1$.
		\end{proof}
	\section{The Archimedean Property}
		\begin{theorem}
			Let $F$ be an archimedean ordered field. Then $F$ is isomorphic to a subfield of the real numbers $\bR$.
		\end{theorem}
	\section{The Least Upper Bound Property and the Importance of the Real Numbers}
		\begin{definition}
			Let $F$ be an ordered field. We say that $F$ has the \tbf{least upper bound property} if every subset of $F$ that has an upper bound in $F$ has a least upper bound in $F$.
		\end{definition}
		\begin{theorem}
			$F$ has the least upper bound property if and only if it has the equivalent "greatest lower bound property", i.e. every subset of $F$ that has a lower bound in $F$ has a greatest lower bound in $F$.
		\end{theorem}
		\begin{theorem}
			Let $F$ be a non-archimedean ordered field. Then $F$ does not have the least-upper-bound property.
		\end{theorem}
		\begin{theorem}
			Every ordered field with the least upper bound property is isomorphic. Therefore the real numbers $\bR$ are, up to isomorphism, the only ordered field with the least upper bound property.
		\end{theorem}
		If you're the kind of person who generally prefers algebra to analysis, you might have always felt unsatisfied by a seeming lack of generality to analysis - why does everyone only ever seem to care about $\bR$? I hope that this theorem finally makes you feel like you have a satisfying answer, as it did for me: Whenever someone makes a definition that explicitly and exclusively concerns the real numbers, they are doing so because they want to make a definition that concerns ordered fields with the least upper bound property - it just so happens that $\bR$ is the only such field!
		\begin{corollary}
			Let $F$ be an ordered field. Then $F$ has the least-upper-bound property if and only if it is archimedean and cauchy complete.
		\end{corollary}
		\subsection{Alternative completeness properties}
		
	\chapter{Topology}
		\section{Metric Spaces and Topological Spaces}
		\begin{definition}
			Let $X$ be a topological space, $x \in X$, and $V \subset X$. We call $V$ a \tit{neighborhood of $x$} if there exists an open set $U \subset V$ such that $x \in U$.
		\end{definition}
		\begin{theorem}
			Let $X$ be a topological space and let $V \subset X$. Then $V$ is open if and only if for every $x \in V$, $V$ is a neighborhood of $x$.
		\end{theorem}
		\begin{proof}
			If $V$ is open then it is trivially a neighborhood of all of its points.
			\newpar
			Assume that $V$ is a neighborhood of all its points. Let $U_x \subset V$ be the necessary open set containing $x \in V$ that makes $V$ a neighborhood of $x$. Then since every $U_x$ is a subset of $V$ we have
			\begin{align*}
				\bigcup_{x \in V} U_x \subset V
			\end{align*}
			and since every $x \in V$ is contained in some $U_x$ we also have
			\begin{align*}
				V \subset \bigcup_{x \in V} U_x
			\end{align*}
			Therefore $V$ is a union of open sets, making it open.
		\end{proof}
		\begin{definition}
			Let $f : X \to Y$ be a function between topological spaces. We call $f$ \tbf{continuous} if the preimage $f^{-1}(U)$ of any open set $U$ is again an open set.
		\end{definition}
		\begin{definition}
			Let $f: X \to Y$ be a function between topological spaces. Let $x \in X$. We call $f$ \tbf{continuous at $x$} if, for any neighborhood $V \subset Y$ of $f(x)$, there exists a neighborhood $U \subset X$ of $x$ such that $f(U) \subset V$.
		\end{definition}
		
		\begin{lemma}
			$f: X \to Y$ is continuous at $x \in X$ if and only if, for every neighborhood $V \subset Y$ of $f(x)$, we have that $f^{-1}(V)$ is a neighborhood of $x$.
		\end{lemma}
		\begin{proof}
			\phantom{}
			\begin{itemize}
				\item[$\Longrightarrow:$] If $f(U) \subset V$, then by the definition of preimages we have $U \subset f^{-1}(V)$. Therefore, since $U$ is a neighborhood of $x$, the superset $f^{-1}(V)$ must be a neighborhood of $x$ as well.
				\item[$\Longleftarrow:$] If $f^{-1}(V)$ is a neighborhood of $x$, then $U = f^{-1}(V)$ already fulfills our definition.
			\end{itemize}
		\end{proof}
		\begin{theorem}
			$f : X \to Y$ is continuous if and only if it is continuous at every point $x \in X$.
		\end{theorem}
		\begin{proof}
			\phantom{}
			\begin{itemize}
				\item[$\Longrightarrow$:] Let $f$ be continuous and let $x \in X$. Then if $V$ is a neighborhood of $f(x)$, there must exist an open set $U$ such that contains $f(x) \in U \subset V$. Then $f^{-1}(U) \subset f^{-1}(V)$ is an open set containing $x$, meaning that $f^{-1}(V)$ is a neigborhood of $x$. Therefore $f$ is continuous at every $x$
				\item[$\Longleftarrow$:] Let $V \subset X$ be open. Then $f^{-1}(V)$ is a neighborhood every $x \in f^{-1}(V)$. Therefore, $f^{-1}(V)$ is open.
			\end{itemize}
		\end{proof}
		\begin{definition}
			Let $f : X \to Y$ be a function between topological spaces. We call $f$ \tbf{sequentially continuous at a point $x$} if, for every sequence $x_n$ such that $\lim_{n \to \infty} x_n = x$, we have
			$\lim_{n \to \infty} f(x_n) = f(x)$. We say the function is \tbf{sequentially continuous} if this condition holds for every point $x \in X$.
		\end{definition}
		\begin{theorem}
			Every continuous function is sequentially continuous.
		\end{theorem}
		\begin{theorem}
			If $X$ is first-countable (and we assume the axiom of choice), then any sequentially continuous function is continuous.
		\end{theorem}
		\begin{corollary}
			A function $f : X \to Y$ from a first-countable space $X$ into any topological space $Y$ is continuous if and only if it is sequentially continuous.
		\end{corollary}
		In particular, continuity and sequential continuity are equivalent for functions between metric spaces.
		\begin{theorem}[\theoremname{Epsilon-Delta-Criterion}]
			Let $f : M \to N$ be a function between metric spaces. Then $f$ is continuous at a point $x \in M$ if and only if for every $\epsilon \in \bR_{> 0}$, there exists a $\delta \in \bR_{> 0}$ such that for all $y \in M$, we have that
			\begin{align*}
				d_M(x,y) < \delta \implies d_N(f(x),f(y)) < \epsilon
			\end{align*}
		\end{theorem}
		This is the standard definition of continuity used in most introductory courses in real analysis.
		Intuitively, it says that a function is continuous if and only if, as $x$ and $y$ get arbitrarily close, $f(x)$ and $f(y)$ also get arbitrarily close. 
		\begin{proof}
			\begin{itemize}
				\item[$\Rightarrow$:] Assume that $f$ is sequentially continuous at a point $x$, but that the given condition doesn't hold. Then there exists an $\epsilon \in \bR_{> 0}$ such that for every $\delta \in \bR_{> 0}$, there exists an $x_\delta \in M$ such that
				\begin{align*}
					d_M(x,x_\delta) \leq \delta, \textnormal{ but } d_N(f(x), f(x_\delta)) \geq \epsilon
				\end{align*}
				Therefore, if we define $\delta_n := \frac{1}{n}$, then the sequence $x_{\delta_n}$ converges to $x$, but the sequence $f(x_{\delta_n})$ doesn't converge to $f(x)$, since $d_N(f(x), f(x_\delta)) \geq \epsilon > 0$.
				\item[$\Leftarrow$:] Let $x_n$ be a sequence with $\lim_{n \to \infty} = x$ which fulfills our condition. We need to show $\lim_{n \to \infty} f(x_n) = f(x)$, meaning that for every $\epsilon \in \bR_{> 0}$, we need to find an $N \in \bN$, such that for all $n \geq N$, we have
				\begin{align*}
					d_N(f(x_n) - f(x)) < \epsilon
				\end{align*}
				by our epsilon-delta condition, this holds for every $x_n$ such that $d(x_n, x) < \delta$. Since $\lim_{n \to \infty} x_n = x$, we can find an $N$ such that this condition is fulfilled for all $n > N$. Therefore does $f(x_n)$ indeed converge to $f(x)$.
			\end{itemize}
		\end{proof}
		\begin{theorem}
			$X$ be a topological space and let $A \subset X$. Then, assuming the discrete topology on $\lr{0,1}$, the characteristic function $\chi_A : X \to \lr{0,1}$ is continuous at a point $x \in X$ if and only if $x \in \tn{int}(A)$ or $x \in \tn{int}(X \setminus A)$.
		\end{theorem}
		\begin{proof}
			\begin{enumerate}
				\item Let $x \in \tn{int}(A)$. Then by definition of the interior of a set there exists an open set $U \subset A$ that contains $x$. Since $U \subset A$, we have $f(U) = \lr{1}$. Therefore, if $V$ is a neighborhood around $f(x) = 1$, then $f^{-1}(V)$ must contain $U$, making it a neighborhood of $x$.
				\item Let $x \in \tn{int}(X \setminus A)$. Then the same argument as before applies, except we have a $U \subset X \setminus A$ with $f(U) = \lr{0}$.
				\item Let $x \in \partial A$ with $x \in A$. Then $V = \lr{1}$ is an open neighborhood of $f(x)$, but $f^{-1}(V) \subset A$. However, since $x$ is on the boundary of $A$, every open set containing $x$ must contain points in $X \setminus A$. Therefore $f^{-1}(V)$ cannot be a neighborhood of $x$.
				\item Let $x \in \partial A$ with $x \in X \setminus A$. Then the same argument applies to $V = \lr{0}$, since $f^{-1}(V)$ cannot contain points in $A$.
			\end{enumerate}
		\end{proof}
		\begin{corollary}
			The characteristic function of the rational numbers is nowhere continuous.
		\end{corollary}
		\begin{theorem}
			\theoremname{(A function continuous at exactly one point)}: The function $f : \bR \to \bR$ with
			\begin{align*}
				f(x) = x \cdot \chi_{\bQ}(x) = \begin{cases}
					x & x \in \bQ\\
					0 & x \notin \bQ
				\end{cases}
			\end{align*}
			is continuous at $0$ and discontinuous at every other point.
		\end{theorem}
		\begin{proof}
			\begin{enumerate}
				\item Let $V$ be a neighborhood of $f(0) = 0$. Then by definition, there must be an $\epsilon > 0$ such that $(-\epsilon,\epsilon) \in V$. Then since $f(x) \leq x$, we have $f^{-1}(y) \geq y$, implying that
				\begin{align*}
					(-\epsilon,\epsilon) \subset f^{-1}((-\epsilon,\epsilon)) \subset f^{-1}(V)
				\end{align*}
				and therefore $f^{-1}(V)$ is a neighborhood of $0$.
				\item Let $x \in \bQ \setminus \lr{0}$. Then, since all irrationals get mapped to zero, the preimage of $(\frac{1}{2}x, \frac{3}{2}x)$ only contains rational numbers and therefore cannot be a neighborhood of $x$.
				\item Let $x \notin \bQ$. Then the preimage of $(-\frac{1}{2}x, \frac{1}{2}x)$ contains $x$, but not any rationals between $x$ and $\frac{1}{2}x$, and therefore cannot be a neighborhood of $x$.
			\end{enumerate}
		\end{proof}
		\begin{theorem}
			\theoremname{(A function only continuous at the irrationals)} \tbf{Thomae's function} $T : \bR \to \bR$, defined as
			\begin{align*}
				T(x) =
				\begin{cases}
					\frac{1}{q} & x \in \bQ, x = \frac{p}{q}, \tn{$p$, $q$ have no common divisors}\\
					0 & x \notin \bQ
				\end{cases},
			\end{align*}
			is discontinuous at every rational number and continuous at every irrational number.
		\end{theorem}
		Thomae's function has many other names - it is also known the \tit{modified Dirichlet function}, the \tit{Riemann function}, or under more whimsical names such as the \tit{popcorn function}, \tit{raindrop function}, \tit{countable cloud function}, or the \tit{Stars over Babylon} (due to John Horton Conway, one of the coolest mathematicians of all time).
		\begin{theorem}
			\theoremname{(A function discontinuous at an arbitrary $F_\sigma$-set)}
			Let $A = \bigcup_{n \in \bN} F_n$ be a countable union of closed sets $F_n$. For any point $x \in A$, let $n(x)$ be the smallest natural number such that $x \in F_{n(x)}$. Then the function $f_A : \bR \to \bR$ defined by
			\begin{align*}
				f_A(x) =
				\begin{cases}
					\frac{1}{n(x)} & x \in A, x \in \bQ\\
					-\frac{1}{n(x)} & x \in A, x \notin \bQ\\
					0 & x \notin A
				\end{cases}
			\end{align*}
			is continuous at every $x \in X \notin A$ and discontinuous at every $x \in A$
		\end{theorem}
		\begin{corollary}
			\theoremname{(Functions continuous at an arbitrary $G_\delta$-set)}
			Since the complement of a $G_\delta$-set is $F_\sigma$, we can use the same construction to construct a function that is continuous at an arbitrary $G_\delta$-subset of $\bR$. 
		\end{corollary}
		\begin{proposition}
			Let $f$ be a function between complete metric spaces. Then the set of discontinuities of $f$ is $F_\sigma$ (meaning it is a countable union of closed sets).
		\end{proposition}
		\begin{corollary}
			There is no function $f : \bR \to \bR$ that is only continuous at the rationals.
		\end{corollary}
		\section{Uniform Spaces}
		Many theorems in analysis require a notion of \tit{uniform convergence}, \tit{uniform continuity}, and so on. These ideas can be easily expressed in a metric space - recall that, for example, a function $f: M \to N$ between metric spaces is uniformly continuous if for every $\epsilon > 0$, a $\delta > 0$ exists such that if $d_M(x,y)  < \delta$, then $d_N(f(x),f(y)) < \epsilon$.
		\newpar
		Meanwhile, we wouldn't be able to refine the definition of continuity like this in a topological space, since the general structure of the neighborhoods of a topological space might vary wildly at different locations in the space - the important quality of a metric space here is that the notion of distance in a metric space can be applied "uniformly" to pairs of points, no matter where they are located. In this section, we want to define a set of spaces more general than metric spaces, but less general than topological spaces, which shares this important property of "uniformity", which will allow us to generalize many useful properties of metric spaces.
		\subsection{Diagonal Uniformity}
		\begin{definition}
			For any set $X$, we denote by $\Delta(X)$ the diagonal $\lr{(x,x) \mid x \in X}$ in $X \times X$.
		\end{definition}
		Our first definition of a \tit{uniform structure} on a set $X$ is based on the observation that in a metric space, $x$ and $y$ are close together if and only if $(x,y)$ is close to $\Delta(X)$.
		\begin{definition}
			For any pair of subsets $U,V$ of $X \times X$ (which by definition can be viewed as relations on $X$), we can extend the notion of function composition to these arbitrary relations by defining $U \circ V$ to be the set
			\begin{align*}
				\lr{(x,y) \in X \times X \mid \exists z \in X : ((x,z) \in V, (z,y) \in U}
			\end{align*}
		\end{definition}
		\begin{definition}
			A \tbf{diagonal uniformity} on a set $X$ is a collection $\cD(X)$ of subsets of $X \times X$, called \tbf{surroundings}, such that:
			\begin{enumerate}
				\item If $D \in \cD$, then $\Delta(X) \subset D$,
				\item If $D_1, D_2 \in \cD$, then $D_1 \cap D_2 \in \cD$,
				\item If $D \in \cD$, then there exists an $E \in \cD$ such that $E \circ E \subset D$,
				\item If $D \in \cD$, then there exists an $E \in \cD$ such that $E^{-1} \subset D$
				\item If $D \in \cD$ and $D \subset E$, then $E \in \cD$.
			\end{enumerate}
			We call a set $X$ equipped with such a structure a \tbf{uniform space}.
		\end{definition}
		\begin{example}
			For any metric space $(M,d)$, the metric $d$ generates a \tit{metric uniformity} by having a surrounding
			\begin{align*}
				D_\epsilon^d = \lr{(x,y) \in M \times M \mid d(x,y) < \epsilon}
			\end{align*}
			for every $\epsilon > 0$. Uniformities that can be generated in this way from metrics are called \tbf{metrizable}.
		\end{example}
		\begin{anmerkung}
			For an arbitrary metric $d$, the uniformity generated by $d$ is identical to the one generated by a scaled version $\lambda d$ (with $\lambda \in \bR^\times$). Therefore different metrics may generate the same uniformity.
		\end{anmerkung}
	\chapter{Topological Vector Spaces}
		\section{Normed Vector Spaces}
		\section{Banach Spaces}
		\section{Hilbert Spaces}
		\section{Topological Vector Spaces}
	\chapter{Differentiation}
		\section{The Frechét Derivative}
		\section{Frechét Spaces}
		\section{The Gateaux Derivative}
	\chapter{Measure Theory}
		\section{Set Algebras}
		\section{Measure Spaces}
		\begin{definition}
			Let $X$ be an uncountable set and $f : X \to [0,\infty]$. Then we define
			\begin{align*}
				\sum_{x \in X} f(x) = \sup_{\abs{F} < \infty} \sum_{x \in F} f(x)
			\end{align*}
		\end{definition}
		It may seem odd that we can sum over an uncountable set by simply summing over the finite sets - this is partly justified by the following lemma:
		\begin{lemma}
			If all terms of a sum are positive, and uncountably many of these terms are non-zero, then the sum diverges.
		\end{lemma}
		\begin{proof}
			Let 
			\begin{align*}
				\sum_{x \in X} f(x) = L \in \bR
			\end{align*}
		Let $S_n = \lr{x \in X : f(x) > \frac{1}{n}}$ for $n \in \bN$. Then we have:
		\begin{align*}
			L &= \sum_{x \in X} f(x)\\
			  &\geq \sum_{x \in S_n} f(x)\\
			  &> \sum_{x \in S_n} \frac{1}{n}\\
			  &= \frac{\abs{S_n}}{n}
		\end{align*}
		So we have $\abs{S_n} < nL$ for all $n \in \bN$. The set of all non-zero terms is given by:
		\begin{align*}
			S = \lr{x \in X \mid f(x) > 0} = \bigcup_{n \in \bN} \lr{x \in X \mid f(x) > \frac{1}{n}}
		\end{align*}
		And since $S$ is a countable union of finite sets, it must be countable. Therefore a finite limit $L$ can't exist if $S$ is uncountable.
		\end{proof}
		\begin{definition}
			If our sum is absolutely convergent, we can also define
			\begin{align*}
				\sum_{x \in X} f(x) = \sum_{x \in X, f(x) \geq 0} f(x) - \sum_{x \in X, f(x) \leq 0} \abs{f(x)}
			\end{align*}
		\end{definition}
		\section{The Lebesque Measure}
	\chapter{Integration}
		\section{The Lebesque Integral}
		\begin{theorem}
			Let $X$ be an arbitrary set. Let $c$ be the counting measure on $\cP(X)$. Let $f : X \to \barR$ be a $c$-measurable function. Then
			\begin{align*}
				\int_X f ~dc = \sum_{x \in X} f(x)
			\end{align*}
		\end{theorem}
		\section{The Bochner Integral}
		
\end{document}